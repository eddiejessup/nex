% This is the plain TeX format that's described in The TeXbook.
% N.B.: A version number is defined at the very end of this file;
%       please change that number whenever the file is modified!
% And don't modify the file unless you change its name:
%       Everybody's "plain.tex" file should be the same, worldwide.

% Unlimited copying and redistribution of this file are permitted as long
% as this file is not modified. Modifications are permitted, but only if
% the resulting file is not named plain.tex.

\catcode`\{=1 % left brace is begin-group character
\catcode`\}=2 % right brace is end-group character
\catcode`\$=3 % dollar sign is math shift
\catcode`\&=4 % ampersand is alignment tab
\catcode`\#=6 % hash mark is macro parameter character
\catcode`\^=7 \catcode`\^^K=7 % circumflex and uparrow are for superscripts
\catcode`\_=8 \catcode`\^^A=8 % underline and downarrow are for subscripts
\catcode`\^^I=10 % ascii tab is a blank space
\chardef\active=13 \catcode`\~=\active % tilde is active
\catcode`\^^L=\active \outer\def^^L{\par} % ascii form-feed is "\outer\par"

\message{Preloading the plain format: codes,}

% We had to define the \catcodes right away, before the message line,
% since \message uses the { and } characters.
% When INITEX (the TeX initializer) starts up,
% it has defined the following \catcode values:
% \catcode`\^^@=9 % ascii null is ignored
% \catcode`\^^M=5 % ascii return is end-line
% \catcode`\\=0 % backslash is TeX escape character
% \catcode`\%=14 % percent sign is comment character
% \catcode`\ =10 % ascii space is blank space
% \catcode`\^^?=15 % ascii delete is invalid
% \catcode`\A=11 ... \catcode`\Z=11 % uppercase letters
% \catcode`\a=11 ... \catcode`\z=11 % lowercase letters
% all others are type 12 (other)

% Here is a list of the characters that have been specially catcoded:
\def\dospecials{\do\ \do\\\do\{\do\}\do\$\do\&%
  \do\#\do\^\do\^^K\do\_\do\^^A\do\%\do\~}
% (not counting ascii null, tab, linefeed, formfeed, return, delete)
% Each symbol in the list is preceded by \do, which can be defined
% if you want to do something to every item in the list.

% We make @ signs act like letters, temporarily, to avoid conflict
% between user names and internal control sequences of plain format.
\catcode`@=11

% INITEX sets up \mathcode x=x, for x=0..255, except that
% \mathcode x=x+"7100, for x = `A to `Z and `a to `z;
% \mathcode x=x+"7000, for x = `0 to `9.
% The following changes define internal codes as recommended
% in Appendix C of The TeXbook:
\mathcode`\^^A="3223 % \downarrow
\mathcode`\^^B="010B % \alpha
\mathcode`\^^C="010C % \beta
\mathcode`\^^D="225E % \land
\mathcode`\^^E="023A % \lnot
\mathcode`\^^F="3232 % \in
\mathcode`\^^G="0119 % \pi
\mathcode`\^^H="0115 % \lambda
\mathcode`\^^I="010D % \gamma
\mathcode`\^^J="010E % \delta
\mathcode`\^^K="3222 % \uparrow
\mathcode`\^^L="2206 % \pm
\mathcode`\^^M="2208 % \oplus
\mathcode`\^^N="0231 % \infty
\mathcode`\^^O="0140 % \partial
\mathcode`\^^P="321A % \subset
\mathcode`\^^Q="321B % \supset
\mathcode`\^^R="225C % \cap
\mathcode`\^^S="225B % \cup
\mathcode`\^^T="0238 % \forall
\mathcode`\^^U="0239 % \exists
\mathcode`\^^V="220A % \otimes
\mathcode`\^^W="3224 % \leftrightarrow
\mathcode`\^^X="3220 % \leftarrow
\mathcode`\^^Y="3221 % \rightarrow
\mathcode`\^^Z="8000 % \ne
\mathcode`\^^[="2205 % \diamond
\mathcode`\^^\="3214 % \le
\mathcode`\^^]="3215 % \ge
\mathcode`\^^^="3211 % \equiv
\mathcode`\^^_="225F % \lor
\mathcode`\ ="8000 % \space
\mathcode`\!="5021
\mathcode`\'="8000 % ^\prime
\mathcode`\(="4028
\mathcode`\)="5029
\mathcode`\*="2203 % \ast
\mathcode`\+="202B
\mathcode`\,="613B
\mathcode`\-="2200
\mathcode`\.="013A
\mathcode`\/="013D
\mathcode`\:="303A
\mathcode`\;="603B
\mathcode`\<="313C
\mathcode`\=="303D
\mathcode`\>="313E
\mathcode`\?="503F
\mathcode`\[="405B
\mathcode`\\="026E % \backslash
\mathcode`\]="505D
\mathcode`\_="8000 % \_
\mathcode`\{="4266
\mathcode`\|="026A
\mathcode`\}="5267
\mathcode`\^^?="1273 % \smallint