% This is the plain TeX format that's described in The TeXbook.
% N.B.: A version number is defined at the very end of this file;
%       please change that number whenever the file is modified!
% And don't modify the file unless you change its name:
%       Everybody's "plain.tex" file should be the same, worldwide.

% Unlimited copying and redistribution of this file are permitted as long
% as this file is not modified. Modifications are permitted, but only if
% the resulting file is not named plain.tex.

\catcode`\{=1 % left brace is begin-group character
\catcode`\}=2 % right brace is end-group character
\catcode`\$=3 % dollar sign is math shift
\catcode`\&=4 % ampersand is alignment tab
\catcode`\#=6 % hash mark is macro parameter character
\catcode`\^=7 \catcode`\^^K=7 % circumflex and uparrow are for superscripts
\catcode`\_=8 \catcode`\^^A=8 % underline and downarrow are for subscripts
\catcode`\^^I=10 % ascii tab is a blank space
\chardef\active=13 \catcode`\~=\active % tilde is active
\catcode`\^^L=\active \outer\def^^L{\par} % ascii form-feed is "\outer\par"

\message{Preloading the plain format: codes,}

% We had to define the \catcodes right away, before the message line,
% since \message uses the { and } characters.
% When INITEX (the TeX initializer) starts up,
% it has defined the following \catcode values:
% \catcode`\^^@=9 % ascii null is ignored
% \catcode`\^^M=5 % ascii return is end-line
% \catcode`\\=0 % backslash is TeX escape character
% \catcode`\%=14 % percent sign is comment character
% \catcode`\ =10 % ascii space is blank space
% \catcode`\^^?=15 % ascii delete is invalid
% \catcode`\A=11 ... \catcode`\Z=11 % uppercase letters
% \catcode`\a=11 ... \catcode`\z=11 % lowercase letters
% all others are type 12 (other)

% Here is a list of the characters that have been specially catcoded:
\def\dospecials{\do\ \do\\\do\{\do\}\do\$\do\&%
  \do\#\do\^\do\^^K\do\_\do\^^A\do\%\do\~}
% (not counting ascii null, tab, linefeed, formfeed, return, delete)
% Each symbol in the list is preceded by \do, which can be defined
% if you want to do something to every item in the list.

% We make @ signs act like letters, temporarily, to avoid conflict
% between user names and internal control sequences of plain format.
\catcode`@=11

% INITEX sets up \mathcode x=x, for x=0..255, except that
% \mathcode x=x+"7100, for x = `A to `Z and `a to `z;
% \mathcode x=x+"7000, for x = `0 to `9.
% The following changes define internal codes as recommended
% in Appendix C of The TeXbook:
\mathcode`\^^@="2201 % \cdot
\mathcode`\^^A="3223 % \downarrow
\mathcode`\^^B="010B % \alpha
\mathcode`\^^C="010C % \beta
\mathcode`\^^D="225E % \land
\mathcode`\^^E="023A % \lnot
\mathcode`\^^F="3232 % \in
\mathcode`\^^G="0119 % \pi
\mathcode`\^^H="0115 % \lambda
\mathcode`\^^I="010D % \gamma
\mathcode`\^^J="010E % \delta
\mathcode`\^^K="3222 % \uparrow
\mathcode`\^^L="2206 % \pm
\mathcode`\^^M="2208 % \oplus
\mathcode`\^^N="0231 % \infty
\mathcode`\^^O="0140 % \partial
\mathcode`\^^P="321A % \subset
\mathcode`\^^Q="321B % \supset
\mathcode`\^^R="225C % \cap
\mathcode`\^^S="225B % \cup
\mathcode`\^^T="0238 % \forall
\mathcode`\^^U="0239 % \exists
\mathcode`\^^V="220A % \otimes
\mathcode`\^^W="3224 % \leftrightarrow
\mathcode`\^^X="3220 % \leftarrow
\mathcode`\^^Y="3221 % \rightarrow
\mathcode`\^^Z="8000 % \ne
\mathcode`\^^[="2205 % \diamond
\mathcode`\^^\="3214 % \le
\mathcode`\^^]="3215 % \ge
\mathcode`\^^^="3211 % \equiv
\mathcode`\^^_="225F % \lor
\mathcode`\ ="8000 % \space
\mathcode`\!="5021
\mathcode`\'="8000 % ^\prime
\mathcode`\(="4028
\mathcode`\)="5029
\mathcode`\*="2203 % \ast
\mathcode`\+="202B
\mathcode`\,="613B
\mathcode`\-="2200
\mathcode`\.="013A
\mathcode`\/="013D
\mathcode`\:="303A
\mathcode`\;="603B
\mathcode`\<="313C
\mathcode`\=="303D
\mathcode`\>="313E
\mathcode`\?="503F
\mathcode`\[="405B
\mathcode`\\="026E % \backslash
\mathcode`\]="505D
\mathcode`\_="8000 % \_
\mathcode`\{="4266
\mathcode`\|="026A
\mathcode`\}="5267
\mathcode`\^^?="1273 % \smallint

% INITEX sets \uccode`x=`X and \uccode `X=`X for all letters x,
% and \lccode`x=`x, \lccode`X=`x; all other values are zero.
% No changes to those tables are needed in plain TeX format.

% INITEX sets \sfcode x=1000 for all x, except that \sfcode`X=999
% for uppercase letters. The following changes are needed:
\sfcode`\)=0 \sfcode`\'=0 \sfcode`\]=0
% The \nonfrenchspacing macro will make further changes to \sfcode values.

% Finally, INITEX sets all \delcode values to -1, except \delcode`.=0
\delcode`\(="028300
\delcode`\)="029301
\delcode`\[="05B302
\delcode`\]="05D303
\delcode`\<="26830A
\delcode`\>="26930B
\delcode`\/="02F30E
\delcode`\|="26A30C
\delcode`\\="26E30F
% N.B. { and } should NOT get delcodes; otherwise parameter grouping fails!

% To make the plain macros more efficient in time and space,
% several constant values are declared here as control sequences.
% If they were changed, anything could happen; so they are private symbols.
\chardef\@ne=1
\chardef\tw@=2
\chardef\thr@@=3
\chardef\sixt@@n=16
\chardef\@cclv=255
\mathchardef\@cclvi=256
\mathchardef\@m=1000
\mathchardef\@M=10000
\mathchardef\@MM=20000

% Allocation of registers

% Here are macros for the automatic allocation of \count, \box, \dimen,
% \skip, \muskip, and \toks registers, as well as \read and \write
% stream numbers, \fam codes, \language codes, and \insert numbers.

\message{registers,}

% When a register is used only temporarily, it need not be allocated;
% grouping can be used, making the value previously in the register return
% after the close of the group.  The main use of these macros is for
% registers that are defined by one macro and used by others, possibly at
% different nesting levels.  All such registers should be defined through
% these macros; otherwise conflicts may occur, especially when two or more
% macro packages are being used at once.

% The following counters are reserved:
%   0 to 9  page numbering
%       10  count allocation
%       11  dimen allocation
%       12  skip allocation
%       13  muskip allocation
%       14  box allocation
%       15  toks allocation
%       16  read file allocation
%       17  write file allocation
%       18  math family allocation
%       19  language allocation
%       20  insert allocation
%       21  the most recently allocated number
%       22  constant -1
% New counters are allocated starting with 23, 24, etc.  Other registers are
% allocated starting with 10.  This leaves 0 through 9 for the user to play
% with safely, except that counts 0 to 9 are considered to be the page and
% subpage numbers (since they are displayed during output). In this scheme,
% \count 10 always contains the number of the highest-numbered counter that
% has been allocated, \count 14 the highest-numbered box, etc.
% Inserts are given numbers 254, 253, etc., since they require a \count,
% \dimen, \skip, and \box all with the same number; \count 20 contains the
% lowest-numbered insert that has been allocated. Of course, \box255 is
% reserved for \output; \count255, \dimen255, and \skip255 can be used freely.

% It is recommended that macro designers always use
% \global assignments with respect to registers numbered 1, 3, 5, 7, 9, and
% always non-\global assignments with respect to registers 0, 2, 4, 6, 8, 255.
% This will prevent ``save stack buildup'' that might otherwise occur.

\count10=22 % allocates \count registers 23, 24, ...
\count11=9 % allocates \dimen registers 10, 11, ...
\count12=9 % allocates \skip registers 10, 11, ...
\count13=9 % allocates \muskip registers 10, 11, ...
\count14=9 % allocates \box registers 10, 11, ...
\count15=9 % allocates \toks registers 10, 11, ...
\count16=-1 % allocates input streams 0, 1, ...
\count17=-1 % allocates output streams 0, 1, ...
\count18=3 % allocates math families 4, 5, ...
\count19=0 % allocates \language codes 1, 2, ...
\count20=255 % allocates insertions 254, 253, ...
\countdef\insc@unt=20 % the insertion counter
\countdef\allocationnumber=21 % the most recent allocation
\countdef\m@ne=22 \m@ne=-1 % a handy constant
\def\wlog{\immediate\write\m@ne} % write on log file (only)

% Here are abbreviations for the names of scratch registers
% that don't need to be allocated.

\countdef\count@=255
\dimendef\dimen@=0
\dimendef\dimen@i=1 % global only
\dimendef\dimen@ii=2
\skipdef\skip@=0
\toksdef\toks@=0

% Now, we define \newcount, \newbox, etc. so that you can say \newcount\foo
% and \foo will be defined (with \countdef) to be the next counter.
% To find out which counter \foo is, you can look at \allocationnumber.
% Since there's no \boxdef command, \chardef is used to define a \newbox,
% \newinsert, \newfam, and so on.

\outer\def\newcount{\alloc@0\count\countdef\insc@unt}
\outer\def\newdimen{\alloc@1\dimen\dimendef\insc@unt}
\outer\def\newskip{\alloc@2\skip\skipdef\insc@unt}
\outer\def\newmuskip{\alloc@3\muskip\muskipdef\@cclvi}
\outer\def\newbox{\alloc@4\box\chardef\insc@unt}
\let\newtoks=\relax % we do this to allow plain.tex to be read in twice
\outer\def\newhelp#1#2{\newtoks#1#1\expandafter{\csname#2\endcsname}}
\outer\def\newtoks{\alloc@5\toks\toksdef\@cclvi}
\outer\def\newread{\alloc@6\read\chardef\sixt@@n}
\outer\def\newwrite{\alloc@7\write\chardef\sixt@@n}
\outer\def\newfam{\alloc@8\fam\chardef\sixt@@n}
\outer\def\newlanguage{\alloc@9\language\chardef\@cclvi}
\def\alloc@#1#2#3#4#5{\global\advance\count1#1by\@ne
  \ch@ck#1#4#2% make sure there's still room
  \allocationnumber=\count1#1%
  \global#3#5=\allocationnumber
  \wlog{\string#5=\string#2\the\allocationnumber}}
\outer\def\newinsert#1{\global\advance\insc@unt by\m@ne
  \ch@ck0\insc@unt\count
  \ch@ck1\insc@unt\dimen
  \ch@ck2\insc@unt\skip
  \ch@ck4\insc@unt\box
  \allocationnumber=\insc@unt
  \global\chardef#1=\allocationnumber
  \wlog{\string#1=\string\insert\the\allocationnumber}}
\def\ch@ck#1#2#3{\ifnum\count1#1<#2%
  \else\errmessage{No room for a new #3}\fi}

% Here are some examples of allocation.
\newdimen\maxdimen \maxdimen=16383.99999pt % the largest legal <dimen>
\newskip\hideskip \hideskip=-1000pt plus 1fill % negative but can grow
\newskip\centering \centering=0pt plus 1000pt minus 1000pt
\newdimen\p@ \p@=1pt % this saves macro space and time
\newdimen\z@ \z@=0pt % can be used both for 0pt and 0
\newskip\z@skip \z@skip=0pt plus0pt minus0pt
\newbox\voidb@x % permanently void box register

% And here's a different sort of allocation:
% For example, \newif\iffoo creates \footrue, \foofalse to go with \iffoo.
\outer\def\newif#1{\count@\escapechar \escapechar\m@ne
  \expandafter\expandafter\expandafter
   \def\@if#1{true}{\let#1=\iftrue}%
  \expandafter\expandafter\expandafter
   \def\@if#1{false}{\let#1=\iffalse}%
  \@if#1{false}\escapechar\count@} % the condition starts out false
\def\@if#1#2{\csname\expandafter\if@\string#1#2\endcsname}
{\uccode`1=`i \uccode`2=`f \uppercase{\gdef\if@12{}}} % `if' is required

% Assign initial values to TeX's parameters

\message{parameters,}

% All of TeX's numeric parameters are listed here,
% but the code is commented out if no special value needs to be set.
% INITEX makes all parameters zero except where noted.

\pretolerance=100
\tolerance=200 % INITEX sets this to 10000
\hbadness=1000
\vbadness=1000
\linepenalty=10
\hyphenpenalty=50
\exhyphenpenalty=50
\binoppenalty=700
\relpenalty=500
\clubpenalty=150
\widowpenalty=150
\displaywidowpenalty=50
\brokenpenalty=100
\predisplaypenalty=10000
% \postdisplaypenalty=0
% \interlinepenalty=0
% \floatingpenalty=0, set during \insert
% \outputpenalty=0, set before TeX enters \output
\doublehyphendemerits=10000
\finalhyphendemerits=5000
\adjdemerits=10000
% \looseness=0, cleared by TeX after each paragraph
% \pausing=0
% \holdinginserts=0
% \tracingonline=0
% \tracingmacros=0
% \tracingstats=0
% \tracingparagraphs=0
% \tracingpages=0
% \tracingoutput=0
\tracinglostchars=1
% \tracingcommands=0
% \tracingrestores=0
% \language=0
\uchyph=1
% \lefthyphenmin=2 \righthyphenmin=3 set below
% \globaldefs=0
% \maxdeadcycles=25 % INITEX does this
% \hangafter=1 % INITEX does this, also TeX after each paragraph
% \fam=0
% \mag=1000 % INITEX does this
% \escapechar=`\\ % INITEX does this
\defaulthyphenchar=`\-
\defaultskewchar=-1
% \endlinechar=`\^^M % INITEX does this
\newlinechar=-1
\delimiterfactor=901
% \time=now % TeX does this at beginning of job
% \day=now % TeX does this at beginning of job
% \month=now % TeX does this at beginning of job
% \year=now % TeX does this at beginning of job
\showboxbreadth=5
\showboxdepth=3
\errorcontextlines=5

\hfuzz=0.1pt
\vfuzz=0.1pt
\overfullrule=5pt
\hsize=6.5in
\vsize=8.9in
\maxdepth=4pt
\splitmaxdepth=\maxdimen
\boxmaxdepth=\maxdimen
% \lineskiplimit=0pt, changed by \normalbaselines
\delimitershortfall=5pt
\nulldelimiterspace=1.2pt
\scriptspace=0.5pt
% \mathsurround=0pt
% \predisplaysize=0pt, set before TeX enters $$
% \displaywidth=0pt, set before TeX enters $$
% \displayindent=0pt, set before TeX enters $$
\parindent=20pt
% \hangindent=0pt, zeroed by TeX after each paragraph
% \hoffset=0pt
% \voffset=0pt

% \baselineskip=0pt, changed by \normalbaselines
% \lineskip=0pt, changed by \normalbaselines
\parskip=0pt plus 1pt
\abovedisplayskip=12pt plus 3pt minus 9pt
\abovedisplayshortskip=0pt plus 3pt
\belowdisplayskip=12pt plus 3pt minus 9pt
\belowdisplayshortskip=7pt plus 3pt minus 4pt
% \leftskip=0pt
% \rightskip=0pt
\topskip=10pt
\splittopskip=10pt
% \tabskip=0pt
% \spaceskip=0pt
% \xspaceskip=0pt
\parfillskip=0pt plus 1fil

\thinmuskip=3mu
\medmuskip=4mu plus 2mu minus 4mu
\thickmuskip=5mu plus 5mu

% We also define special registers that function like parameters:
\newskip\smallskipamount \smallskipamount=3pt plus 1pt minus 1pt
\newskip\medskipamount \medskipamount=6pt plus 2pt minus 2pt
\newskip\bigskipamount \bigskipamount=12pt plus 4pt minus 4pt
\newskip\normalbaselineskip \normalbaselineskip=12pt
\newskip\normallineskip \normallineskip=1pt
\newdimen\normallineskiplimit \normallineskiplimit=0pt
\newdimen\jot \jot=3pt
\newcount\interdisplaylinepenalty \interdisplaylinepenalty=100
\newcount\interfootnotelinepenalty \interfootnotelinepenalty=100

% Definitions for preloaded fonts

\def\magstephalf{1095 }
\def\magstep#1{\ifcase#1 \@m\or 1200\or 1440\or 1728\or 2074\or 2488\fi\relax}

% Fonts assigned to \preloaded are not part of "plain TeX",
% but they are preloaded so that other format packages can use them.
% For example, if another set of macros says "\font\ninerm=cmr9",
% TeX will not have to reload the font metric information for cmr9.

\message{fonts,}

\font\tenrm=cmr10 % roman text
\font\preloaded=cmr9
\font\preloaded=cmr8
\font\sevenrm=cmr7
\font\preloaded=cmr6
\font\fiverm=cmr5

\font\teni=cmmi10 % math italic
\font\preloaded=cmmi9
\font\preloaded=cmmi8
\font\seveni=cmmi7
\font\preloaded=cmmi6
\font\fivei=cmmi5

\font\tensy=cmsy10 % math symbols
\font\preloaded=cmsy9
\font\preloaded=cmsy8
\font\sevensy=cmsy7
\font\preloaded=cmsy6
\font\fivesy=cmsy5

\font\tenex=cmex10 % math extension

\font\preloaded=cmss10 % sans serif
\font\preloaded=cmssq8

\font\preloaded=cmssi10 % sans serif italic
\font\preloaded=cmssqi8

\font\tenbf=cmbx10 % boldface extended
\font\preloaded=cmbx9
\font\preloaded=cmbx8
\font\sevenbf=cmbx7
\font\preloaded=cmbx6
\font\fivebf=cmbx5

\font\tentt=cmtt10 % typewriter
\font\preloaded=cmtt9
\font\preloaded=cmtt8

\font\preloaded=cmsltt10 % slanted typewriter

\font\tensl=cmsl10 % slanted roman
\font\preloaded=cmsl9
\font\preloaded=cmsl8

\font\tenit=cmti10 % text italic
\font\preloaded=cmti9
\font\preloaded=cmti8
\font\preloaded=cmti7

\message{more fonts,}
\font\preloaded=cmu10 % unslanted text italic

\font\preloaded=cmmib10 % bold math italic
\font\preloaded=cmbsy10 % bold math symbols

\font\preloaded=cmcsc10 % caps and small caps

\font\preloaded=cmssbx10 % sans serif bold extended

\font\preloaded=cmdunh10 % Dunhill style

\font\preloaded=cmr7 scaled \magstep4 % for titles
\font\preloaded=cmtt10 scaled \magstep2
\font\preloaded=cmssbx10 scaled \magstep2

\font\preloaded=manfnt % METAFONT logo and dragon curve and special symbols

% Additional \preloaded fonts can be specified here.
% (And those that were \preloaded above can be eliminated.)

\let\preloaded=\undefined % preloaded fonts must be declared anew later.

\skewchar\teni='177 \skewchar\seveni='177 \skewchar\fivei='177
\skewchar\tensy='60 \skewchar\sevensy='60 \skewchar\fivesy='60

\textfont0=\tenrm \scriptfont0=\sevenrm \scriptscriptfont0=\fiverm
\def\rm{\fam\z@\tenrm}
\textfont1=\teni \scriptfont1=\seveni \scriptscriptfont1=\fivei
\def\mit{\fam\@ne} \def\oldstyle{\fam\@ne\teni}
\textfont2=\tensy \scriptfont2=\sevensy \scriptscriptfont2=\fivesy
\def\cal{\fam\tw@}
\textfont3=\tenex \scriptfont3=\tenex \scriptscriptfont3=\tenex
\newfam\itfam \def\it{\fam\itfam\tenit} % \it is family 4
\textfont\itfam=\tenit
\newfam\slfam \def\sl{\fam\slfam\tensl} % \sl is family 5
\textfont\slfam=\tensl
\newfam\bffam \def\bf{\fam\bffam\tenbf} % \bf is family 6
\textfont\bffam=\tenbf \scriptfont\bffam=\sevenbf
\scriptscriptfont\bffam=\fivebf
\newfam\ttfam \def\tt{\fam\ttfam\tentt} % \tt is family 7
\textfont\ttfam=\tentt

% Macros for setting ordinary text
\message{macros,}

\def\frenchspacing{\sfcode`\.\@m \sfcode`\?\@m \sfcode`\!\@m
  \sfcode`\:\@m \sfcode`\;\@m \sfcode`\,\@m}
\def\nonfrenchspacing{\sfcode`\.3000\sfcode`\?3000\sfcode`\!3000%
  \sfcode`\:2000\sfcode`\;1500\sfcode`\,1250 }

\def\normalbaselines{\lineskip\normallineskip
  \baselineskip\normalbaselineskip \lineskiplimit\normallineskiplimit}

\def\^^M{\ } % control <return> = control <space>
\def\^^I{\ } % same for <tab>

\def\lq{`} \def\rq{'}
\def\lbrack{[} \def\rbrack{]}

\let\endgraf=\par \let\endline=\cr

\def\space{ }
\def\empty{}
\def\null{\hbox{}}

\let\bgroup={ \let\egroup=}

% In \obeylines, we say `\let^^M=\par' instead of `\def^^M{\par}'
% since this allows, for example, `\let\par=\cr \obeylines \halign{...'
{\catcode`\^^M=\active % these lines must end with %
  \gdef\obeylines{\catcode`\^^M\active \let^^M\par}%
  \global\let^^M\par} % this is in case ^^M appears in a \write
\def\obeyspaces{\catcode`\ \active}
{\obeyspaces\global\let =\space}

\def\loop#1\repeat{\def\body{#1}\iterate}
\def\iterate{\body \let\next\iterate \else\let\next\relax\fi \next}
\let\repeat=\fi % this makes \loop...\if...\repeat skippable

\def\thinspace{\kern .16667em }
\def\negthinspace{\kern-.16667em }
\def\enspace{\kern.5em }

\def\enskip{\hskip.5em\relax}
\def\quad{\hskip1em\relax}
\def\qquad{\hskip2em\relax}

\def\smallskip{\vskip\smallskipamount}
\def\medskip{\vskip\medskipamount}
\def\bigskip{\vskip\bigskipamount}

\def\nointerlineskip{\prevdepth-1000\p@}
\def\offinterlineskip{\baselineskip-1000\p@
  \lineskip\z@ \lineskiplimit\maxdimen}

\def\topglue{\nointerlineskip\vglue-\topskip\vglue} % for top of page
\def\vglue{\afterassignment\vgl@\skip@=}
\def\vgl@{\par \dimen@\prevdepth \hrule height\z@
  \nobreak\vskip\skip@ \prevdepth\dimen@}
\def\hglue{\afterassignment\hgl@\skip@=}
\def\hgl@{\leavevmode \count@\spacefactor \vrule width\z@
  \nobreak\hskip\skip@ \spacefactor\count@}

\def~{\penalty\@M \ } % tie
\def\slash{/\penalty\exhyphenpenalty} % a `/' that acts like a `-'

\def\break{\penalty-\@M}
\def\nobreak{\penalty \@M}
\def\allowbreak{\penalty \z@}

\def\filbreak{\par\vfil\penalty-200\vfilneg}
\def\goodbreak{\par\penalty-500 }
\def\eject{\par\break}
\def\supereject{\par\penalty-\@MM}

\def\removelastskip{\ifdim\lastskip=\z@\else\vskip-\lastskip\fi}
\def\smallbreak{\par\ifdim\lastskip<\smallskipamount
  \removelastskip\penalty-50\smallskip\fi}
\def\medbreak{\par\ifdim\lastskip<\medskipamount
  \removelastskip\penalty-100\medskip\fi}
\def\bigbreak{\par\ifdim\lastskip<\bigskipamount
  \removelastskip\penalty-200\bigskip\fi}

\def\line{\hbox to\hsize}
\def\leftline#1{\line{#1\hss}}
\def\rightline#1{\line{\hss#1}}
\def\centerline#1{\line{\hss#1\hss}}

\def\rlap#1{\hbox to\z@{#1\hss}}
\def\llap#1{\hbox to\z@{\hss#1}}

\def\m@th{\mathsurround\z@}
\def\underbar#1{$\setbox\z@\hbox{#1}\dp\z@\z@
  \m@th \underline{\box\z@}$}

\newbox\strutbox
\newbox\strutbox
